\documentclass{article}
\usepackage[utf8]{inputenc}
\usepackage[margin = 0.8in]{geometry}
\usepackage{graphicx}
\usepackage{amsmath, amssymb}
\usepackage{caption}
\usepackage{subcaption}
\usepackage{multirow}
\usepackage{multicol}
\usepackage{booktabs}
\usepackage{bm}
\setlength{\columnsep}{.75cm}
\usepackage{float}
\usepackage{graphicx}
\usepackage{stfloats}
\usepackage{hyperref}
\graphicspath{ {./images/} }
\setcounter{MaxMatrixCols}{15}

\renewcommand\refname{References}
\restylefloat{table}


\title{Autonomous Operation for Grocery Delivery on an Urban Sidewalk }
\author{Keith Chester, Bob DeMont}
\date{February 1, 2022}

\begin{document}
\maketitle


\begin{multicols}{2}
\section*{Abstract}
The purpose of the project is to simulate successful autonomous navigation of a grocery delivery vehicle (Grocerybot) along sidewalks in a busy urban environment from store to delivery address on ground level.

\section*{Introduction}
Last mile delivery has been a growing issue with the increase in e-commerce over at least the last 8 years with each year growing from 15-30\% from the year before \ref{Ecom}.  Since much of this final delivery is accomplished via vehicles, the has been an accompanying call for reduced emissions.  A McKinsey study estimates that this could lead to a 25\% increase in CO2 emission in cities\ref{Emiss}.  Most recently, the COVID pandemic with its isolation mandates have led to even greater demand for delivery of not only goods but also everyday essentials such as meals and groceries.  This needed to be accomplished with minimal human contact while fewer humans were available due to quarantines and employee availability.  For grocery delivery, the need for fastest route planning is also necessary due to the presence of perishables.
Cheng et al \ref{} have shown efficient curb detection to set the limits of the Grocerybot's path.

\section*{Goals}
Our goal is to create planning and execution algorithms to permit autonomous navigation and avoidance of both static and dynamic obstacles while following sidewalk rules (crosswalks etc)

\section*{Proposed Methods}
We plan to explore various navigation, collision detection and avoidance methods implemented in Python to successfully navigate around static obstacles, avoid dynamic obstacles  and reach the destination in a simulated urban environment in Gazebo.

\section*{Expected Results}
We expect to create a Gazebo-based simulation to navigate from the store to a nearby, ground level delivery address via sidewalks.  Along the way, we plan to encounter static obstacles such as fire hydrants, garbage cans and boxes.  We also expect to encounter moving obstacles like people or other Grocerybots.  Through this we will avoid collisions to arrive at the delivery destination while following sidewalk rules.

\section*{Proposed Schedule}
We propose development along the following schedule:

\begin{center}
\begin{tabular}{|cc|}
\hline
Milestone & Date  \\
\hline
Proposal  &  Feb 1 \\
Gazebo learning  &  Feb 1-18 \\
Status Update  &  Feb 21 \\
Sample environment & Feb 28\\
Planning algorithm & Mar 14\\
Obstacle avoidance algorithms & Apr 11\\
Final Writeup and Presentation & Apr 25-29\\
Final Submission  &  May 2 \\
\hline
\end{tabular}
%\caption{Schedule}
%\label{table:Sched}
\end{center}

\section*{Division of Labor}
At this point we plan extensive and equal collaboration on all aspects of the project to include coding, algorithm development, testing and submissions.

\label{References}

\bibliographystyle{abbrv}
\begin{thebibliography}{10}
\bibitem{Ecom} https://www.statista.com/statistics/379046/worldwide-retail-e-commerce-sales/
\bibitem{Emiss}https://www.mckinsey.com/industries/travel-logistics-and-infrastructure/our-insights/efficient-and-sustainable-last-mile-logistics-lessons-from-japan
\bibitem{Kocs}
M. Kocsis, J. Buyer, N. Sußmann, R. Zöllner and G. Mogan, "Autonomous Grocery Delivery Service in Urban Areas," 2017 IEEE 19th International Conference on High Performance Computing and Communications; IEEE 15th International Conference on Smart City; IEEE 3rd International Conference on Data Science and Systems (HPCC/SmartCity/DSS), 2017, pp. 186-191, doi: 10.1109/HPCC-SmartCity-DSS.2017.24.
\bibitem{Mcheng}
M. Cheng, Y. Zhang, Y. Su, J. M. Alvarez and H. Kong, "Curb Detection for Road and Sidewalk Detection," in IEEE Transactions on Vehicular Technology, vol. 67, no. 11, pp. 10330-10342, Nov. 2018, doi: 10.1109/TVT.2018.2865836.
\bibitem{AHM}
\bibitem{MLT}



\end{thebibliography}

\end{multicols}

\end{document}
